\documentclass{beamer}

\title{\large Solution of Optimization Problems in Mathcad Prime\textsuperscript{\tiny\textregistered}}
\author{Fərid Fərəcli \and Orxan Məmmədli \and Tayfur Tağıyev}
\institute{Azerbaijan State Oil and Industry University}

\usetheme{Frankfurt}
\usepackage{fontspec}

\begin{document}

\maketitle

\begin{frame}
\frametitle{Table of Contents}
\tableofcontents
\end{frame}

\section{Introduction}

\begin{frame}{What is Optimization}
\begin{itemize}
\item Definition: “Optimization is the process of finding the best solution from all feasible solutions.”
\vspace{10pt}
\item Core elements: Objective function, decision variables, constraints
\vspace{10pt}
\item Simple examples: Minimizing cost, maximizing efficiency
\vspace{10pt}
\item Real-life analogy (e.g., finding the fastest route on a map)
\end{itemize}
\end{frame}

\begin{frame}{Types of Optimization Problems}
\begin{itemize}
\item Linear vs. Nonlinear
\item Unconstrained vs. Constrained
\item Single-objective vs. Multi-objective
\item Continuous vs. Discrete (Integer Optimization)
\end{itemize}
\end{frame}

\begin{frame}{Applications of Optimization}
\begin{itemize}
\item Engineering design (e.g., minimizing material use)
\item Operations research (e.g., supply chain optimization)
% \item Finance (e.g., portfolio optimization)
\item Machine learning (e.g., cost function minimization)
\end{itemize}
\end{frame}

\begin{frame}{Overview of Mathcad Prime as a Computational Tool}
\begin{itemize}
\item Intuitive math-like interface
\item Symbolic and numerical solving capabilities
\item Easy visualization of functions
\item Integration with engineering calculations
\item Documentation and calculations in one place
\end{itemize}
\end{frame}

\section{Optimization Capabilities in Mathcad Prime}

\begin{frame}{Key features for optimization}
\begin{itemize}
\item WYSIWYG interface: Equations look as they are written in math
\item Dynamic recalculation with parameter changes
\item Support for symbolic and numerical calculations
\item Built-in optimization tools ($minimize()$, $maximize()$, $solve block$)
\item Plotting capabilities for function visualization
\end{itemize}
\end{frame}

\begin{frame}{Built-in Optimization Functions}
\begin{itemize}
\item $minimize(f, x)$: Finds the value of $x$ that minimizes function $f$
\item $maximize(f, x)$: Finds the value of $x$ that maximizes function $f$
\item \textbf{Solve block}: More flexible method for complex problems, including constraints
\item Requires initial guesses
\end{itemize}
\end{frame}

\begin{frame}{Mathcad Syntax and Variable Definition Essentials}
\begin{itemize}
\item Variables are defined using $:=$
\item Functions are defined similarly to programming, e.g., $f(x) := x^2 + 3$
\item Units can be used and converted
\item Importance of defining guess values before using $minimize()$ or $maximize()$
\end{itemize}
\end{frame}

\begin{frame}{Constraints in Optimization Problems}
\begin{itemize}
\item Constraints can be equations ($=$) or inequalities ($\leq$, $\geq$)
\item Constraints are used inside solve blocks
\item Example: Subject to $x \geq 0$, $x + y = 10$
\item Use $Find(variable)$ inside solve block to obtain result
\end{itemize}
\end{frame}

\section{Solving Different Types of Optimization Problems}

\begin{frame}{Unconstrained Optimization Problems}
\begin{itemize}
\item No restrictions or limitations on variable values
\item Common for simple mathematical functions
\item Use $minimize(f, x)$ or $maximize(f, x)$
\item Requires a good initial guess
\end{itemize}
\end{frame}

\begin{frame}{Valediction}
\begin{center}
\Huge Thank you for attention!
\end{center}
\end{frame}

\end{document}
